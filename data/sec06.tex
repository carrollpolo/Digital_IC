\section{总结与展望/ConclusIon And RecommendAtIons}

随着摩尔定律逐渐逼近物理极限,传统单片式集成电路(Monolithic IC)在性能提升、成本控制和制造良率等方面面临严峻挑战。Chiplet(芯粒)架构作为一种突破性设计范式,通过将复杂系统分解为多个功能独立、可复用的小芯片模块,并利用先进封装技术实现高密度互连,有效实现了异构集成、工艺解耦与成本优化,已成为后摩尔时代集成电路发展的关键技术路径。

本文围绕Chiplet架构的核心要素,从五个维度系统性地探讨了其关键技术体系。首先,是设计目标与应用介绍介绍 Chiplet 架构的基本概念。接着在分解层面,介绍了Chiplet划分(Die分区)的基本原则,包括功能模块化、工艺节点解耦、良率优化与通信开销权衡,揭示了模块化设计如何提升设计灵活性与经济性。其次,在连接层面,深入分析了2.5D/3D封装、硅中介层(Silicon Interposer)、硅桥(Silicon Bridge)等先进互连结构,并对比了UCIe等开放互连标准的技术特性,强调了高带宽、低延迟、高能效互连对系统性能的关键影响。再次,在计算与数据协同的内存存储与传输层面,探讨了Chiplet系统中的内存架构设计,重点介绍了高带宽内存(HBM)的集成方式、分布式缓存一致性协议(如MESI、MOESI扩展)以及多层次存储体系的构建策略,以缓解“内存墙”问题。最后,在功耗与散热管理层面,阐述了Chiplet系统中局部热点管理等关键技术,以及介绍EDA厂商如何使用仿真模拟设计低功耗与散热良好的系统,用以确保系统在高性能运行下的可靠性与稳定性。
