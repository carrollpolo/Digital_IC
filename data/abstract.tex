
% 中英文摘要和关键字
% 中文摘要
\renewcommand{\abstractname}{摘要} % 将 abstract 标题改为“摘要”
\begin{abstract}
Chiplet(芯粒)架构是一种新兴的半导体设计模式,它旨在应对摩尔定律放缓和先进工艺制造成本日益高昂的挑战。本论文系统地介绍了Chiplet架构的基本概念、特性及其在后摩尔时代集成电路设计的新范式等领域的广泛应用。以下是该篇论文的具体结构划分。

第 1 章绪论:设计目标与应用介绍介绍 Chiplet 架构的基本概念和特性(异构集成、模块化),以及 Chiplet技术的发展和应用领域(设计目标与典型应用场景);
第 2 章从“分解”的观点出发,讲述 Chiplet 划分(Die 分区)的基本理论,介绍功能分离、工艺解耦、良率优化和通信考量等概念;
第 3 章从“连接”的观点出发,讲述 2.5D/3D 封装、硅中介层和硅桥等典型微系统互连技术的一般理论和特性,给出常用的互连标准(如 UCIe)、带宽密度和相应能效参数等;
第 4 章采用“计算”与“数据”相结合的方法,讲述 Chiplet 系统的内存存储策略,其中包括高带宽内存(HBM)集成、分布式缓存一致性协议和存储层次结构等;
第 5 章介绍 Chiplet 系统中功耗与散热管理的工作原理和应用。


\end{abstract}

% 中文关键词
\noindent \textbf{关键词:} 
Chiplet (芯粒) 架构,
异构集成,
先进封装,
互连技术 / Chiplet 互连 ,
Die 分区 / 模块化设计,
高性能计算,
UCIe,
功耗与散热管理
\addcontentsline{toc}{section}{摘要} % 将摘要添加到目录中

\; 

\renewcommand{\abstractname}{Abstract} % 恢复为默认的“Abstract”


\begin{abstract}
The Chiplet architecture is an emerging semiconductor design paradigm designed to address the challenges of slowing Moore's Law and the increasing cost of advanced process manufacturing. This paper systematically introduces the basic concepts and characteristics of the chiplet architecture, as well as its broad applications in areas such as the new paradigm of integrated circuit design in the post-Moore era. The following is the detailed structure of this paper.

Chapter 1: Introduction: Design Goals and Applications introduces the basic concepts and characteristics of the chiplet architecture (heterogeneous integration and modularity), as well as the development and application areas of chiplet technology (design goals and typical application scenarios).
Chapter 2, starting from the perspective of "decomposition," discusses the basic theory of chiplet partitioning (die partitioning), introducing concepts such as functional separation, process decoupling, yield optimization, and communication considerations.
Chapter 3, starting from the perspective of "connectivity," discusses the general theory and characteristics of typical microsystem interconnect technologies such as 2.5D/3D packaging, silicon interposers, and silicon bridges, and presents commonly used interconnect standards (such as UCIe), bandwidth density, and corresponding energy efficiency parameters.
Chapter 4, taking a combined "compute" and "data" approach, describes memory storage strategies for chiplet systems, including high-bandwidth memory (HBM) integration, distributed cache coherence protocols, and storage hierarchies.
Chapter 5 introduces the working principles and applications of power consumption and thermal management in chiplet systems.


\end{abstract}

% 英文关键词
\noindent \textbf{Keywords:}
Chiplet Architecture,
Heterogeneous Integration,
Advanced Packaging,
Interconnect Technology/Chiplet Interconnect,
Die Partitioning/Modular Design,
High-Performance Computing (HPC),
UCIe,
Power and Thermal Management

\addcontentsline{toc}{section}{Abstract} % 将摘要添加到目录中
